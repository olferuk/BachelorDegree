\section*{Заключение}
\addcontentsline{toc}{section}{Заключение}

В рамках данной работы была разработан готовый программный продукт: библиотека для indoor-локализации для платформы iOS. Согласно выдвинутым требованиям:
\begin{enumerate}
    \item
    Произведен детальный анализ различных алгоритмов трилатерации, выделены их сильные и слабые стороны;
    \item
    В библиотеку включены несколько реализаций алгоритмов, выделенных в результате анализа, а так же предоставлена возможность использовать их комбинации;
    \item
    Работа алгоритмов максимально оптимизирована как по скорости работы, так и по объему занимаемой памяти;
    \item
    Библиотека легко подключается в iOS-приложение, а ее первоначальная настройка занимает минимальный объем кода.
\end{enumerate}

Однако даже на данном этапе можно выделить направления для дальнейших улучшений. 

Прежде всего, можно передавать показания акселерометра, магнитометра и гироскопа в качестве управляющих воздействий в фильтр Калмана, и, как результат, во время движения пользователя в большей степени доверять показаниям ИНС, а во время его остановок - показаниям маяков.

Так как навигация внутри помещений может быть трактована как подзадача нахождения положения пользователя в целом, возможна лучшая интеграция с сервисами навигации GPS и <<ГЛОНАСС>>.

