\section{Глоссарий}

LBS (“Location Based Service”) – сервис, основанный на местоположении.

GPS (“Global Positioning System”) – глобальная система позиционирования.

LBA (“Location Based Advertising”) – геоконтекстная реклама.

RFID (“Radio Frequency Identification”) – радиочастотная идентификация.

NFC (“Near Field Communication”) – ближняя бесконтактная связь.

BLE (“Bluetooth Low Energy”) – беспроводная технология Bluetooth с низким энергопотреблением.

SDK (“Standard Developer Kit”) – стандартный набор разработчика.

API (“Application Programming Interface”) – интерфейс программирования приложений.

CMS (“Content Management System”) – система управления содержимым.

REST (“Representational State Transfer”) – передача репрезентативного состояния.

DoS (“Denial of Service”) – отказ в обслуживании.

AGA (“Adaptive Geometric Algorithm”) – адаптивный геометрический алгоритм.

Спуффинг маячков (“beacon spoofing”) – тип хакерской атаки, заключающийся в выставлении поддельного маячка, настроенного с параметрами маяков некоторой существующей группы маяков.

Пиггибэкинг маячков (“beacon piggybacking”) – тип хакерской атаки, основанный на использовании параметров маячков, определенных в одном приложении, в стороннем приложении, принадлежащем третьим лицам. Сам же термин “piggybacking” с английского можно перевести как «несанкционированное проникновение вслед за зарегистрированным пользователем».

Fingerprints (с англ. буквально «отпечатки пальцев») – способ нахождения положения пользователя на основе сравнения ряда измерений расстояний до маяков с набором эталонных измерений, определенных для известных локаций. \\

Фильтр частиц (“particle filter”) – метод трилатерации, в котором в процессе генерации частицы (точки в двумерном пространстве) все лучше приближают истиное положение пользователя, а наименее значимые исключаются из рассмотрения.

