\section*{Приложения}
\addcontentsline{toc}{section}{Приложения}

\lstset{basicstyle=\footnotesize \ttfamily}
\lstset{language=C, breakindent=40pt, breaklines=true}

\subsection*{Приложение 1. Реализация алгоритма, основанного на поиске силового центра}

\begin{lstlisting}
double * calculateUserPositionPowerCenter(double *xs, double *ys, double *accs) {

    double *result = (double *)malloc(sizeof(double) * DIMENSIONS);

    // k[n] = ( B[n].x^2 + B[n].y^2 - dist[n]^2 )/2
    double *ks = (double *)malloc(sizeof(double) * MIN_BEACONS);
    for (size_t i = 0; i < MIN_BEACONS; ++i) {
        ks[i] = (xs[i]*xs[i] + ys[i]*ys[i] - accs[i]*accs[i])/2.0;
    }
    
    // D = | x1-x2  y1-y2 | = (x1-x2)*(y2-y3) - (y1-y2)*(x2-x3)
    //     | x2-x3  y2-y3 |
    double d = (xs[0] - xs[1])*(ys[1] - ys[2]) - (ys[0] - ys[1])*(xs[1] - xs[2]);
    
    if (fabs(d) < 1e-7) {
        result[0] = -1;
        return result;
    }
    
    // X = | k1-k2  y1-y2 | = (k1-k2)*(y2-y3) - (y1-y2)*(k2-k3)
    //     | k2-k3  y2-y3 |
    double x = (ks[0] - ks[1])*(ys[1] - ys[2]) - (ys[0] - ys[1])*(ks[1] - ks[2]);
    
    // Y = | x1-x2  k1-k2 | = (x1-x2)*(k2-k3) - (k1-k2)*(x2-x3)
    //     | x2-x3  k2-k3 |
    double y = (xs[0] - xs[1])*(ks[1] - ks[2]) - (ks[0] - ks[1])*(xs[1] - xs[2]);
    
    result[0] = x/d;
    result[1] = y/d;
    
    return result;
}
\end{lstlisting}


\subsection*{Приложение 2. Реализация алгоритма, основанного на пересечении сфер}

\begin{lstlisting}
double * calculateUserPositionSphereIntersection(double *xs, double *ys, double *accs) {
    double *result = (double *)malloc(sizeof(double) * DIMENSIONS);

    double temp = (xs[1] - xs[0])*(xs[1] - xs[0]) + (ys[1] - ys[0])*(ys[1] - ys[0]);
    double exx = (xs[1] - xs[0]) / sqrt(temp);
    double exy = (ys[1] - ys[0]) / sqrt(temp);
    
    double p3p1x = xs[2] - xs[0];
    double p3p1y = ys[2] - xs[0];
    
    double ival = exx * p3p1x + exy*p3p1y;
    
    double p3p1i = (xs[2] - xs[0] - exx)*(xs[2] - xs[0] - exx) + (ys[2] - ys[0] - exy)*(ys[2] - ys[0] - exy);
    
    double eyx = (xs[2] - xs[0] - exx) / sqrt(p3p1i);
    double eyy = (ys[2] - ys[0] - exy) / sqrt(p3p1i);
    
    double d = sqrt(temp);
    
    double jval = (eyx * p3p1x) + (eyy * p3p1y);
    
    double xval = (accs[0]*accs[0] - accs[1]*accs[1] + d*d) / (2*d);
    double yval = (accs[0]*accs[0] - accs[2]*accs[2] + ival*ival + jval*jval)/(2*jval) - (ival/jval)*xval;
    
    result[0] = xs[0] + exx*xval + eyx*yval;
    result[1] = ys[0] + exy*xval + eyy*yval;
    
    return result;
}
\end{lstlisting}


\subsection*{Приложение 3. Реализация адаптивного геометрического алгоритма}

\begin{lstlisting}

#define MAX_ITERATIONS  500
#define Points          struct point *
#define enoughPoints    7
#define step            0.1f

/**
 *  Struct represents a beacon or a point
 */
struct point {
    double x;
    double y;
    double r;
};

/**
 *  Checks whether the point belongs to all three circles
 *
 *  @param x      x
 *  @param y      y
 *  @param points Beacons
 *
 *  @return 1 if true, 0 otherwise
 */
int isPointBelongToAllCircles(double x, double y, Points points) {
    int belongs = 1;
    for (size_t i = 0; i < MIN_BEACONS; ++i) {
        double dist = getDistance(x, y, points[i].x, points[i].y);
        if (dist > (points[i].r + 1e-2)) {
            belongs = 0;
            break;
        }
    }
    return belongs;
}

/**
 *  Returns the centroid coordinate of given points
 *
 *  @param points Array of points
 *  @param size   Its size
 *
 *  @return Centroid coordinates
 */
Points getCenter(Points points, size_t size) {
    Points center = (Points )malloc(sizeof(struct point));
    center->x = 0;
    center->y = 0;
    
    for (size_t i = 0; i < size; ++i) {
        center->x += points[i].x;
        center->y += points[i].y;
    }
    center->x /= size;
    center->y /= size;
    
    return center;
}

/**
 *  Calculates all circle-circle intersections and return an array of resulting point
 *
 *  @param pointA Circle A
 *  @param pointB Circle B
 *  @param cnt    Reference to the resulting array's size
 *
 *  @return Array of points
 */
Points getCircleCircleIntersection(Points pointA, Points pointB, int *cnt) {
    *cnt = 0;
    double r1 = pointA->r;
    double r2 = pointB->r;
    double p1x = pointA->x;
    double p1y = pointA->y;
    double p2x = pointB->x;
    double p2y = pointB->y;
    double d = getDistance(p1x, p1y, p2x, p2y);
    
    // if too far away, or self contained - can't be done
    if ((d >= (r1 + r2)) || (d <= fabs(r1 - r2))) {
        return 0;
    }
    
    double a = (r1*r1 - r2*r2 + d*d)/(2*d);
    double h = sqrt(r1*r1 - a*a);
    double x0 = p1x + a*(p2x - p1x)/d;
    double y0 = p1y + a*(p2y - p1y)/d;
    double rx = -(p2y - p1y)*(h/d);
    double ry = -(p2x - p1x)*(h/d);
    
    *cnt = 2;
    Points result = (Points )malloc(*cnt * sizeof(struct point));
    result[0].x = x0 + rx;
    result[0].y = y0 - ry;
    result[1].x = x0 - rx;
    result[1].y = y0 + ry;
    return result;
}

/**
 *  Returns an array of circle intersections
 *
 *  @param points Beacons
 *  @param size   Size of the array
 *  @param cnt    Reference to the size of the resulting array
 *
 *  @return Array of points
 */
Points getIntersectionPoints(Points points, size_t size, int *cnt) {
    size_t e = 20;
    Points result = (Points )malloc(e * sizeof(struct point));
    
    size_t indexToAdd = 0;
    for (size_t i = 0; i < size; ++i) {
        for (size_t j = 1; j < size; ++j) {
            int cnt = 0;
            Points intersects = getCircleCircleIntersection(&points[i], &points[j], &cnt);
            if (cnt > 0) {
                for (size_t k = 0; k < cnt; ++k) {
                    result[indexToAdd++] = intersects[k]; 
                }
            }
        }
    }
    
    return result;
}

/**
 *  Selects only common points from the circles intersection
 *
 *  @param points Intersection points
 *  @param size   Size of that array *  @param beacons The given beacons and their accuracies
 *  @param cnt    Reference to the resulting array's count
 *
 *  @return Array of points
 */
Points getCommonPoints(Points points, size_t size, Points beacons, int *cnt) {
    Points result = (Points)malloc(sizeof(struct point) * enoughPoints);
    
    int k = 0;
    for (size_t i = 0; i < size; ++i) {
        if (!isPointBelongToAllCircles(points[i].x, points[i].y, beacons)) {
            continue;
        }
        
        result[k].x = points[i].x;
        result[k].y = points[i].y;
        result[k].r = points[i].r;
        ++k;
    }
    *cnt = k;
    
    return result;
}

/**
 *  Mutates three arrays of coordinates into one array of points
 *
 *  @param xs   xs
 *  @param ys   ys
 *  @param accs accs
 *
 *  @return Array of points
 */
Points createPoints(double *xs, double *ys, double *accs) {
    Points points = (Points )malloc(MIN_BEACONS * sizeof(struct point));
    for (size_t i = 0; i < MIN_BEACONS; ++i) {
        points[i].x = xs[i];
        points[i].y = ys[i];
        points[i].r = accs[i];
    }
    return points;
}

/**
 *  Enlarges the beacons' accuracies proportionally
 *
 *  @param beacons Array of beacons
 */
void enlargeAccuracies(Points beacons) {
    for (size_t i = 0; i < MIN_BEACONS; ++i) {
        beacons[i].r *= (1 + step);
    }
}

//************************************************************************

double * calculateUserPositionEpta(double *xs, double *ys, double *accs) {
    double *result = (double *)malloc(sizeof(double) * DIMENSIONS);

    Points beacons = createPoints(xs, ys, accs);

    int iterations = 0;
    while (1) {
        if (++iterations > MAX_ITERATIONS) {
            return result;
        }
        
        int intersectionCount = 0;
        Points intersections = getIntersectionPoints(beacons, MIN_BEACONS, &intersectionCount);
        
        if (intersectionCount == 0) {
            return result;
        }
        
        int commonCount = 0;
        Points common = getCommonPoints(intersections, intersectionCount, beacons, &commonCount);
        
        if (commonCount == 2 || commonCount == 3) {
            Points center = getCenter(common, commonCount);
            result[0] = center[0].x;
            result[1] = center[0].y;
            return result;
        }
        
        enlargeAccuracies(beacons);
    }
    
    return result;
}
\end{lstlisting}


\subsection*{Приложение 4. Начальная настройка подключенной библиотеки}

\lstset{language=[Objective]C}

\begin{lstlisting}
#import <BeaconLocation/BeaconLocation.h>
@import CoreGraphics;
// ...
@property (nonatomic, strong) BeaconLocation *library;
// ...
@interface MyClass: NSObject <BeaconLocationDelegate>
// ...
- (void)init {
  self = [super init];
  if (self) {
    //init
    _library = [[BeaconLocation alloc] initWithUUIDString:@"12345678-1234-0000-4321-876543210000" 
                                               identifier:@"My region"];

    //beacons
    [_library.floor addBeaconWithMajor:0 minor:0 inPosition:CGPointMake(1, 2)];
    [_library.floor addBeaconWithMajor:0 minor:1 inPosition:CGPointMake(2, 3)];
    [_library.floor addBeaconWithMajor:0 minor:2 inPosition:CGPointMake(1, 3)];

    //method
    [_library.processor setAlgorithm:AlgorithmTypeSphereIntersection];

    //delegate
    _library.delegate = self;
  }
  return self;
}

- (void)onUpdateUserPostion:(CGPoint)position {
  // do fancy stuff
}
\end{lstlisting}