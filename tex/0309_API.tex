\subsection{Анализ предоставляемого API}

Для indoor-навигации Apple предоставляет библиотеку CoreLocation. Она содержит несколько основных сущностей, использованных в работе:

\begin{itemize}
    \item
    \texttt{CLBeacon}. Класс, инкапсулирующий как идентификаторы маяка (uuid, major, minor), так и информацию, используемую при ранжировании (rssi, accuracy, proximity).
    \item
    \texttt{CLBeaconRegion}. Класс, содержащий идендификаторы группы маяков (uuid, [major, [minor]]). С его помощью можно удобно разделять группы маячков, находящихся, например, на разных этажах, и работать с каждой группой по отдельности.
    \item
    \texttt{CLLocationManager}. Центральный класс API, который может отслеживать маячки в указанном регионе (\texttt{CLBeaconRegion}), рассылая событиями как факты входа или выхода в зону действия маячков (\textit{mo\-ni\-to\-ring}), так и прием пакетов для определения расстояния до источника (\textit{rang\-ing}).
\end{itemize}

Таким образом, Apple представила не только формат на BLE-маячки iBeacon, ставший уже широко распространенным, но и удобные средста для работы с ними.