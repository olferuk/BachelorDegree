\subsection{Анализ предоставляемого API}

Для indoor-навигации Apple предоставляет библиотеку CoreLocation. Она содержит несколько основных сущностей, использованных в работе:

\begin{itemize}
    \item
    CLBeacon. Класс, инкапсулирующий как идентификаторы маяка (uuid, major, minor), так и информацию, используемую при ранжировании (rssi, accuracy, proximity).
    \item
    CLBeaconRegion. Класс, содержащий идендификаторы группы маяков (uuid, [major, [minor]]). С его помощью можно удобно разделять группы маячков, находящихся, например, на разных этажах, и работать с каждой группой по отдельности.
    \item
    CLLocationManager. Центральный класс API, который может отслеживать маячки в указанном регионе (CLBeaconRegion), рассылая событиями как факты входа или выхода в зону действия маячков (monitoring), так и прием пакетов для определения расстояния до источника (ranging).
\end{itemize}

Таким образом, Apple представила не только формат на BLE-маячки iBeacon, ставший уже широко распространенным, но и удобные средста для работы с ними.